\documentclass[margin, line]{resume}

\usepackage[cm]{fullpage}

\usepackage[usenames, dvipsnames]{color}
\usepackage[usenames, dvipsnames]{xcolor}
\usepackage{graphicx}
\usepackage{tikz}
\usepackage{wrapfig}
\usepackage{multicol}
\usepackage{marvosym}
\usepackage{hyperref}
\usepackage{pifont}
\usepackage{helvetica}
\usepackage{enumitem}

\definecolor{urlcolor}{rgb}{0.2, 0.2, 0.4}
\definecolor{headingcolor}{rgb}{0.0, 0.0, 0.0}
\definecolor{verticalcolor}{rgb}{0.92, 0.92, 0.92}
\hypersetup{colorlinks, urlcolor=urlcolor, linkcolor=urlcolor}

\frenchspacing
\oddsidemargin -.5in
\evensidemargin -.5in
\textwidth=6.0in
\itemsep=0in
\parsep=0in
\pagestyle{empty}

\newenvironment{list1}{
  \begin{list}{\ding{113}}{%
      \setlength{\itemsep}{0in}
      \setlength{\parsep}{0in} \setlength{\parskip}{0in}
      \setlength{\topsep}{0in} \setlength{\partopsep}{0in}
      \setlength{\leftmargin}{0.17in}}}{\end{list}}

\newenvironment{list4}{
%%  \begin{list}{}{%
	\begin{list}{\small{\ding{110}}}{%
      \setlength{\itemsep}{0in}
      \setlength{\parsep}{0in} \setlength{\parskip}{0in}
      \setlength{\topsep}{0in} \setlength{\partopsep}{0in}
      \setlength{\leftmargin}{0.2in}}}{\end{list}}

\newlist{list3}{itemize}{1}
\setlist[list3]{label=\textbullet, labelindent=-.3in, leftmargin=0pt, labelsep=.15in }
	
\newlist{list2}{itemize}{1}
\setlist[list2]{label=\enspace, labelindent=-.5in, leftmargin=0pt, labelsep=.25in }

\begin{document}

\name{\huge{\textsc{Jaap Suter}} \vspace*{.03in}}

\begin{resume}

\vspace{.05in}
\begin{tabular}{@{}p{2in}p{4in}}
+1 604 313 5227 \it{(cell)} 					          & 1738 Parker Street         		\\
\href{mailto:contact@jaapsuter.com }{contact@jaapsuter.com} & Vancouver BC \hspace{1mm}V5L 2K8  \\
\href{http://www.jaapsuter.com}{www.jaapsuter.com}     	 & Canada                            \\
\end{tabular}

\section{\sc Experience}

{\bf Sabbatical} - Vancouver, Canada \hfill {\bf April 2011 - Present}
\\
\begin{list3}
	\item Experimenting with a custom \href{https://github.com/JaapSuter/Ten18#readme}{CLR host and custom C\#/C++ interop layer} to allow use of DirectX 11 alongside .NET's Async Await, without compromising
		 framerate or input latency.
	\item Developing a camera calibration method that uses fringe patterns on a flat-screen monitor rather than checkerboard shots to streamline the process.
	\item Build a simple \href{https://github.com/JaapSuter/Pentacorn#readme}{Structured Light 3D Scanner} in C\# \it{(XNA, Emgu, and DirectShow)} using a projector and camera.	
\end{list3}

{\bf Electronic Arts} - Burnaby, Canada \hfill {\bf November 2004 - April 2011}

Technical Director for Core \& Infrastructure. Responsible for the core runtime libraries used by nearly all EA games on all platforms, a variety of tools used by artists, QA, and producers at different EA studios worldwide, and the entire stack of package distribution, configuration management, and project build technologies necessary for effective sharing of cross-platform technology.

Alongside, major initiatives included:
\\
\begin{list3}
\item{Drove the mobile strategy for central technology, porting a majority of our base libraries to iPhone/iOS, porting our build infrastructure to support Mac OSX and Xcode development, and ensuring Android, WebOS, and others got similar attention.}
\item{Created the EASharp project, allowing C\# programming for game consoles (PS3, X360, Wii). Wrote the ahead-of-time MSIL to \href{http://llvm.org/}{LLVM} compiler, targeting a custom .NET runtime. Build a C\# to Console toolchain around it, ported the \href{http://www.hpl.hp.com/personal/Hans_Boehm/gc/}{Boehm collector}, replaced it with a custom garbage collector, and ported the initial subset of the Base Class Libraries.}
\item{Job Manager.}
\end{list3}

{\bf Next Level Games} - Vancouver, Canada \hfill {\bf October 2004 - October 2005}

Senior Software Engineer, member of the Super Mario Strikers team.
Responsible for the architecture, design and implementation of major
subsystems and guidance and mentoring of junior engineers.

{\bf Electronic Arts} - Burnaby, Canada \hfill {\bf September 2002 - September 2004}

Software engineer, part of the Fifa team, responsible for many
aspects of several games, including databases, rendering, Xbox SKUs,
artist tools, and more.

{\bf Overloaded Games} - Amsterdam, The Netherlands \hfill {\bf March 2002 - September 2002}

Principal software engineer on three PocketPC \& Symbian games, as well as main contributor to the 
\emph{MobileCore} base libraries on which these games were built.

{\bf Davilex Games} - Houten, The Netherlands \hfill {\bf September 1999 - January 2002}

Software Engineer in the R\&D department, wrote the software rasterizer and 3D engine used in garden and interior design products, and the end-to-end lighting pipeline for several racing games.

\pagebreak

\section{\sc Education}
{\bf Twente University}, Enschede, The Netherlands \hfill {\bf September 1998 - September 2002}

B.Sc. in Computer Science

\section{\sc Skills}
\begin{list3}
\item Strong and well-rounded software engineer, delivers quality and elegance.
\item Not afraid of math.
\item Comfortable multi-paradigm programmer, be it object oriented, generic, functional, meta or generative programming.
\item Strong advocate of test driven, agile and common sense development.
\item Demonstrated ability to lead and mentor other engineers
\item Proven track record architecting and maintaining physical aspects of large scale multi-language cross-platform projects.
\item Very comfortable with concurrency/parallelism, and solid experience designing portable multi-core architectures.
\item Expert C++ programmer, comfortable with the STL, Boost, and C++0x.
\item Expert C\# programmer, experienced with Linq, and Reactive Extensions, and Async Await.
\item Experienced C and assembly programmer, comfortable writing code for embedded devices with limited memory and performance.
\item Dabbles in: Python, Java, Lisp, ARM/Neon and SPU Assembly.
\item Platform experience: Win32/64, PS3 (PPU/SPU), Xbox 360, Wii, iPhone, Xbox, PS2, Gamecube, GBA
\end{list3}

\section{\sc Publications}
Suter J. (2003) - \textsl{Geometric Algebra Primer} - Introduction to Clifford Algebra, available\\ online at \url{http://www.jaapsuter.com/geometric-algebra/}.

\section{\sc \href{http://www.mobygames.com/developer/sheet/view/developerId,94489/}{Credits}}
\begin{multicols}{3}
\begin{list2}
    \item Burnout Paradise (2009)
    \item NBA Street (2007)
    \item NFS: Carbon (2006)    
    \item Super Mario Strikers (2005)
    \item FIFA Soccer 2005 (2004)
    \item UEFA Euro 2004 (2004)
    \item FIFA Soccer 2004 (2003)
    \item The Sims Bustin' Out (2003)
    \item US Racer (2002)
    \item Snowboard Jam (2002)
    \item Nim \& Lost Garden (2002)
    \item Magnets (2002)
    \item US Racer (2002)
    \item Wonen (2000)
    \item DaviTuin 3D (2000)    
\end{list2}
\end{multicols}

\end{resume}
\end{document}
