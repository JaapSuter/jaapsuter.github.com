\documentclass[twocolumn]{paper}
\begin{document}

\title{On Newsgroups}
\author{Jaap Suter}
\maketitle

\section{Caveat}

I originally started this document with a larger intended audience.
The very same problems are being addressed by the CTO-X group as
part of the transparency, open-source and documentation discussions.
EATech and Frank Russel are also addressing some of the issues
herein. Alas, in the interest of time of getting this out the door,
I quickly wrapped the end up with a Criterion focus and we'll take
it from there.

\section{Introduction}

Communication is a bottleneck in most things we do at EA. Hence,
it's understandable we keep trying to find better ways to talk to
each other. In the past few weeks there has been a discussion about
the need for newsgroups in Criterion. In essence this is a great
idea, but executed poorly it'll end up on the long list of failed
attempts to improve communication.

Therefore, we need to identify what we are trying to accomplish and
learn from past lessons to know what we have to avoid. We can then
make recommendations on how to move forward.

\section{Why We Need It}

Newsgroups are only a means to an end. Unless we understand what the
end is and we all agree on it, we can't successfully deploy a new
communication tool.

\subsection{Active Information}

Many EA employees write great emails on a daily basis. These are
then send to a small number of people either trough a mail group or
through a random list of addresses the author chooses himself.
Assuming the email has great public content --we obviously write a
great deal of poor emails and emails intended only for a select
group--, there are two problems with this approach.

\begin{itemize}
    \item Because we want to make sure we include everybody, we end
    up sending the email to too many people. The spam effect ensues.
    \item Because we are afraid of the spam effect, we are afraid to
    send it to people that might take interest too. So some people
    miss out on the information.
\end{itemize}

These two problems come from a larger underlying problem; emails are
active pieces of information. When an email is sent, it has the urge
to penetrate into your inbox. You stop whatever it is you were
doing, and have a sudden urge to read and often reply too. And
because they are disruptive, people restrict their audience.

You lose both ways; when you do send an email you disturb the
receiving end, when you don't send it people miss out on important
information.

\subsection{Passive Information}

The solution is simple; selectively move away from active
information and make it passive. Put the onus on the reader to find
information. Assuming information is easily accessible --see below--
people can adapt their reading patterns around the work they are
doing, instead of having their work patterns dictated by the flow of
their inbox.

An often heard objection against passive information is that people
won't take the effort to find information. This is mostly a cultural
phenomenon; when people are used to getting information delivered to
them it takes a while to adapt. Obviously important information can
and should still be send out actively.

A second excuse for people to ignore the passive information is poor
accessibility. The technology has to make it easy to gain access.
Active information, in the form of email and instant messengers, has
reached the point where it's ubiquitous, uniform, and idiot proof.
Passive information is lacking in this regard; there are NNTP
newsgroups, archived mailing lists, many different HHTP-based forum
systems, different WIKI implementations and an enormous amount of
different clients to access these different delivery mechanisms.

These problems of passive information are discussed below. Let's
direct our attention to the benefits for a moment.

\subsection{More Information}

As mentioned above, there is a crucial advantage to passive
information; the receiving end can decide when and what to read.
Arguably email allows you to do this, but in practice all incoming
email is read and often replied to in a schedule that your mail
client dictates.

Because the reader has control over his consumption patterns,
contributors don't have to worry about the spam effect. One result
is that more information is written and the potential audience is a
lot larger. The second result is that multiple streams of
information can be combined into one to enable peer review and cross
pollination.

\subsubsection{Peer Review}

Imagine two engineers discussing a particular algorithm they want to
use in their project. Often, they'd be discussing this amongst
themselves over email. Nobody in the company would even know of this
discussion, let alone be able to contribute to it.

Now imagine a newsgroup, mailing list or forum that is accessible
everywhere through EA. Suppose the afore mentioned discussion
happened on there instead, which is ok since there is no risk of the
spam effect. Now everybody glancing over the newsgroup sees this
topic being discussed; they can decide to ignore it, they can read
it and learn from the information, or possibly they have actual
experience with the algorithm enabling them to contribute.

Instead of two engineers having a private discussion, there are now
2000 engineers; 1900 of which don't care, 95 that read the thread
and learn something from it, and five that contribute to a solution
that is better than what the two engineers alone could have come up
with.

\subsubsection{Cross Pollination}

The advantage of peer review is a horizontal one. People are exposed
to discussions on the same level and topics that they are
experienced with, allowing them to give feedback and contribute.

Cross pollination is a similar advantage, but it works vertically.
In a company largely comprised of knowledge workers, we have to
acknowledge that most people can provide valuable input far beyond
their specific area of expertise.

Imagine two engineers having a discussion that appears to be a
rendering problem but effectively revolves around an architectural
design flaw. We can easily imagine an audio programmer or technical
artist joining the discussion because of their expertise or even
just common sense.

Many technical issues have a more abstract underlying problem that
most people at EA can sensibly talk about. Passive delivery of
information allows us to tap this free potential around the company.

\section{Lessons}

The ideas in this document aren't new. In fact, they've been around
since the dawn of electronic communication. Yet somehow, many
attempts to enable passive communication within the company are not
working. Unless we understand why this is, we risk adding another
failed attempt.

\subsection{Transitions}

We aren't interested in passive information systems to discuss new
material. We look into it because we think it'll enable us to talk
about the same things in different and better ways. Therefore,
unless the new form of communication replaces the old one, many
people will continue to use the old methods. The early adopters of
the new structure realize they can't reach their intended audience
anymore and quickly they are forced to go back to the old system.

To enable adoption and success, we need to make quick switch from
the old system to the new system. In EA's case this means disabling
many existing email groups and installing bounce emails that
redirect the user to the new source.

Such a transition strategy would raise objections across EA. To
mitigate that, any new solution has to deliver ease of use and
traffic. If the new mechanism fails either of these two, the system
will fail.

\subsection{Ease Of Use}

As mentioned earlier, email and instant messages are idiot proof.
They are so easy to use that the supporting mechanisms have moved to
the background completely, allowing us to focus on the content.

Existing passive communication mechanisms have a lot of work to do
in this area. Newsgroups have been around for a long time but
unfortunately Outlook doesn't come with a newsreader by default.
Outlook Express does but not everybody like to use it. Having to use
a completely different client all together turns many people away.
Not all mailing-lists and newsgroups come with a web-interface too
and when they do it's never quite they way most users want it. Some
people continue to prefer active messages, by email, regardless of
the passive structure underneath.

Ease of use is absolutely crucial for success. To get there, we need
to deliver the following

\begin{itemize}
    \item Proper threading, such that related posts can form a
    collapsible hierarchy.
    \item Access through Outlook and Outlook Express.
    \item An easy to find web-interface with a trivial URL.
    \item Lightweight access through this web-interface for both reading and
    posting, Google or GMANE style.
    \item The ability to send an email somewhere to post on the
    newsgroup.
    \item The ability to receive newsgroup messages in your inbox anyway.
    \item A reliable archive with easy and powerful search
    capabilities.
\end{itemize}

\subsection{Traffic}

Newsgroups can be incredibly easy to use, but unless there is actual
traffic on them there is no point in using them. Traffic is what
enables peer review and cross pollination to work and that's
precisely why we are looking into passive communication in the first
place.

\subsubsection{Start Big}

The worst idea for a newsgroup is to limit its use to a specialized
topic. People are afraid of the spam effect and decide to create a
number of small communities so avoid intruding on each others space,
thereby completely neglecting the notion that newsgroup information
is passive. It's precisely the point of a newsgroup that one can
cater to a large community. Readers can easily skim and ignore
threads they don't care about and focus on the pieces of information
they do think is interesting.

A newsgroup should start as big as possible. When experience shows
that a community beings to emerge that warrants splitting off, a new
group can easily be created. The idea is to start consolidated and
organically grow a sustainable number of communities.

Interestingly enough, this is the exact opposite of what EATech has
been doing recently with the small web-based forums dedicated to
each particular technology and share-point. Traffic will continue to
be low, and this will limit their chance of success.

\subsubsection{Public Access}

Another great way to reduce the amount of traffic in a community is
to restrict access. Again, it goes against the idea of peer review
and cross pollination. By limiting the number of users that can read
and post on a newsgroup we'll miss out on the greatest benefits.
Instead, we should cater to the largest audience possible; all of
Electronic Arts.

One objection is that not all discussions are meant for people
outside of Criterion. The first counter-argument would be; why is
that? Not being able to share information says more about the source
of the information than about the receiving end. If we want to be
more transparent as a company and use this to generate more buy-in
and leverage design feedback from other teams, EA worldwide
accessible newsgroups would be the best first start.

The second counter-argument is that nothing stops us from still
having an internal conversation. Many non-technical issues are best
discussed with only a few people anyway and these continue to work
well using email.

\subsubsection{Maintenance}

The third way to lose traffic is to neglect maintenance. Many people
assume that a community works by setting up the technical
infrastructure and then let it run itself. Nothing is farther from
the truth. A community can only work because a group of dedicated
people invest time and effort. Closer investigation of any
successful open-source or other online community shows there are
always a number of key people that keep traffic flowing, orchestrate
threads, moderate posts, reply more than average, redirect posts to
ping other people, etcetera.

Nourishing a newsgroup takes time and effort and while there is no
need to put this responsibility on particular people --they usually
emerge automatically-- there does need to be a recognition that this
is time well spend.

\section{Recommendations}

Based on the above observations and assumptions, I'd like to
recommend the following.
\begin{itemize}
    \item Start a Criterion-Development newsgroup for all technical
    discussions.
    \item Eventually we might split off a Criterion-Users group that is
    intended for questions about existing technology; versus the other
    that is intended for new development.
    \item Make the newsgroup available to everybody
    within EA worldwide for both reading and writing.
    \item License an Outlook newsgroup plug-in so anybody can read
    the newsgroups from within their default mail client.
    \item License NTTP-to-HTTP technology so people can read and
    post to the newsgroup from a browser
    \item Dedicate resources to enable email-based posting and
    reading.
    \item Dedicate resources to enable powerful and easy searching
    and archiving.
    \item Instantly remove all appropriate mailing-groups and have those addresses bounce with the announcement email. Avoid a
    deprecation strategy.
    \item Write a detailed announcement email explaining our
    motivation and rationales. Include some basic guidelines on
    successful large community posting (use of [tech], etc.) as well
    as a guide for the Outlook plugin and directions to other
    clients and the web interface.
    \item Encourage use of the newsgroup by redirecting appropriate emails you
    receive to the newsgroup.
\end{itemize}

\end{document}
