\documentclass[twocolumn]{paper}
\begin{document}

\title{Developing A Team}
\author{Jaap Suter}
\maketitle

\begin{abstract}
The upcoming transition to a next generation of consoles will expose
the problems of our industry more than ever. The time to revisit our
strategies and drastically change our practices is now.

Through the years I have accumulated many theories on how game
development can be done better. Most ideas herein come from other
people, and I have merely taken it upon myself to present them
together. I provide only underlying ideas and postpone practical
suggestions to later articles. Because of my background and the
nature of the problems in our industry, this paper will have a
strong engineering focus.

The first section identifies four values that underly any successful
development effort. The second section offers some key observations
based on common sense that somehow can't seem to effect change in
our industry. The third section provides some practical suggestions
on how a modern team would approach a development.

\end{abstract}

\section{Values}

Several years ago, Matt Brown wrote an article called Funatic Values
and Ranting. It explained how underlying any group effort lie
several fundamental ideas that determine the level of success the
group can attain. I agree with Matt that explicitly stating these
values is important. It is easy to forget them and measure
everything solely by its immediate practical features. Such
ignorance allows a team to drift unnoticeably away from the rewards
that people look for in work and life, until it is too late.

Once we know what core values motivate us, we know what to look for
in our day to day tasks. Be it hiring, programming, scheduling,
animating, testing or anything else; we can ask ourselves how it
aligns with the fundamental ideas that drive us.

Motivation is the best catalyst for productivity. What follows is a
summary of the values that motivate me. The similarities with Matt
Brown's values are apparent, and I recommend the reader to get a
copy of his essay.

\subsection{Integrity}

Of the four values, integrity is the most important one. A team can
only work at peak performance when all members can trust and respect
each other. Without this, the need for defense systems arises
resulting in hostility and denial of responsibility. Inevitably this
hampers communication, slowing down progress as a result.

An environment built on trust allows for risk taking, which is the
primary factor that drives our business. We make money when we
successfully follow a path that our competitors didn't dare to go
on. With risk comes failure and only teams that accept this are able
to come up with the gems that inspire us all.

\subsection{Excitement}

Most software engineers could make more money writing content
management systems, database queries or web servers. Instead, some
decide to work in the games industry because they are looking for
something more. It allows them to work on cutting edge technology,
with creative, talented and passionated people, creating a product
that entertains people around the world.

Our desire to work in the games industry can be measured by our
excitement. The more excited we are, the more we realize how much we
enjoy our work. And it is when we enjoy our work that we can be most
productive.

When the excitement dies, the joy disappears and with it our
productivity. Mundane tasks are rarely useful and there is often a
way to transform it into a more efficient, more effective and more
exciting task.

Excitement or lack thereof can result in feedback loops in either
direction. Exciting people tend to stimulate each other to do great
things. Negativity creates an environment that drags itself further
down.

Aim to work on exciting things. Let lack of excitement be an
indicator of poor strategy and strive to transform.

\subsection{Pride}

It is a basic human desire to aim for excellence when compared to
others. This applies on an individual level as well as a team and
company level. Individuals, teams and companies that try to do
better than the rest come out ahead. Taking pride in your own work,
in team results and in company progress is a huge stimulus for
success.

With pride comes accountability. Those individuals and teams that
hold themselves accountable for their work deliver higher quality
and won't use lack of responsibility as an excuse for failure.

Not all environments are conducive to pride. Several factors greatly
affect how proud we can be about what we do.

\subsubsection{Identity}

Unless there is an identity associated with a person or a team,
there is nothing to attach pride to. People need form and space to
stick onto their labour. Great work without personality is void. We
need to see who and where it came from to bestow the praise a person
deserves.

Likewise, teams need an identity and location too. The motivational
power of allowing a team to grow itself a non-generic descriptive
name should not be underestimated. Allowing people to own the area
they work in provides a sense of ownership. Instead of scattering
development through a company across multiple game, art and
technology teams, we should encourage proximity and collaboration.
Allowing people to shape their physical work environment the way
they see fit helps form a team identity.

\subsubsection{Challenge}

When we work on simple problems we feel unchallenged and see our
excitement and pride dwindling. Solving difficult problems allows us
to grow, get excited and be proud. Greater challenges offer greater
rewards on a personal and team level.

If the path to a solution is covered with mundane obstacles that
should have been solved already, it takes away our focus and we
waste time instead of being dedicated to the real solution.

Whenever we work on a task that feels easy or we find ourselves
sidetracked too much, we need to consider three things. Is it
possible to automate part of the process? Is it possible to
outsource? Is there a higher level of abstraction that can solve
many easy problems at once?

We can be successful when we tackle the challenges our competitors
missed. We can generate excitement and pride when we solve hard
problems. We can be more efficient when we automate and outsource
the easy and mundane.

\subsubsection{Peers}

Peer recognition is an incredible motivator, especially when our
colleagues are doing work equally or more impressive than what we do
ourselves. If all team members are the cream of the crop, it creates
a shared feeling of pride. People are excited they are part of the
team. They understand this privilege comes with high expectations
and make sure they live up to them.

A simple way to ruin a star team is to add underachievers. The
obvious consequence is that some people work significantly harder
than others, causing a rift. The less obvious and often neglected
consequence is that the best team members lose their drive to
perform better than everybody else. Compared to the team average
they already do exceptionally well at medium capacity. This is not
just a waste of talent, it also removes all sense of pride and
excitement. Great people have a desire to work with greater people,
and will search elsewhere if their current position can't offer
that.

Assembling a great team does not imply a sole need for senior people
with many years of experience. In fact, some of the most motivated
and smartest members can come straight out of school. Often young
additions can add the fresh perspective that a team of experienced
people needs to keep an open mind. Not everybody is equally
productive, nor is everybody willing to work long hours for extended
periods of time. The key understanding is that all team members aim
for excellence within their capabilities and that each reaches
beyond his or her grasp.

\subsection{Growth}

Whether or not we enjoy our work is just one factor that determines
our success. Equally important, if not more, is that we grow as
people and as a team; not in numbers but in the ways we work,
collaborate, develop, interact and live. Any time a lesson is
learned and shared, no time has been wasted.

However talented and skilled all team members may be, we must all
realize that our primary role is as a student. When we are convinced
there are other people better at one thing or another, we open
ourselves up to learn from them. Whenever people interact, all
parties should aim to come out more capable, more skilled, more
knowledgable and emotionally stronger. It is only when we humbly
place ourselves in the vulnerable position of the student that we
improve ourselves and reach higher levels.

Simultaneously, everyone has an important role as a mentor. Each
team member has skills that other team members have not yet learned.
Instead of keeping those skills to ourselves for fear of wasting
time or to safeguard a reputation, we ought to dedicate time and
energy to spread these skills as widely as possible. The entire team
and company benefits when multiple people can do what earlier only
one person could accomplish. Furthermore, teaching is an incredibly
rewarding experience that allows the teacher to further refine his
skills and grow on a personal and emotional level.

Next to informal, practical and short term mentor-student
relationships, people have a need for long term structured career
trajectories. Many companies ask their employees where they see
themselves in five years, but don't use this information to create
strategies. This hinders growth of internal staff. It also urges
people to look elsewhere for ways to accomplish their career
objectives. Structured trajectories force us to think about our
goals, then focus on them and truly grow. This keeps us challenged,
proud and excited.

\section{Observations}

This section discusses some key observations from which modern
development strategies can be derived. Some of these understandings
might read as common sense, yet it is surprisingly easy to point out
many processes and environments that are based on assumptions
contradictory to the following observations.

\subsection{On Intelligence}

All software engineers know that the productivity of bad, mediocre,
good and great engineers increases exponentially. In other words, a
good programmer can be four times as useful as a mediocre one. A
great programmer can do the work of ten bad ones. Unfortunately,
human resources and financial planners do not always understand this
point and easily replace a productive programmer with two poor
programmers for half the price.

Granted, productivity is not easily measured whereas salaries and
job descriptions are. This makes the latter obvious tools for
quantitative headcount planning. There is a strong need for
qualitative headcount planning which takes intelligence and
productivity into account and places this above positions and
salaries.

Smarter engineers are more productive by themselves, allowing the
team size to decrease. This in itself provides a potential cost
benefit. Additionally there is a reduced need for management and
communication channels can run smoother, in turn providing more cost
savings.

Small teams of skilled people are more agile and adaptable than
large teams. They are capable of dealing with changes in
requirements, schedules and features more quickly. They can be more
productive than lowest common denominator teams. Most importantly,
they bond together more easily thereby catalyzing excitement and
productivity.

\subsection{On Game Development}

Traditionally game developers have distanced themselves from what
the rest of the software industry did. Time after time we reinvented
the wheel or publicly ridiculed the inefficient processes,
languages, datastructures and algorithms that the \emph{boring}
industry used. Meanwhile, other sectors have learned from each other
and successfully developed applications that easily rival the best
games in terms of complexity, performance, scope and robustness.

Our projects have long ago caught up with other forms of software
development, and the time has come for the people and processes to
catch up too. Once we realize that some of the hardest problems in
our projects are also being solved by other applications, we can
look at them with an open mind and learn.

Game development is still a unique blend of creativity, technology
and passion that few other industries can offer. Nonetheless, even
game developers are in the business of software development and as
such we need to study the lessons that other sectors have learned
before us.

\subsection{On Complexity}

The scope of projects in the games industry is growing at
lightspeed. More than ever is the need to manage complexity
apparent. Teams that don't change their development style to cope
with the size of their products will run into a disaster of which we
already saw faint symptoms on current generation projects. Modern
development requires two major paradigm shifts.

First, the scope of the game and its subsystems is sufficiently
large that no single design effort can adequately prepare for the
implementation of any feature. Traditional software engineering
models that design and build large systems in isolation and then
deploy will cause significant integration headaches when they are
used on any large project. Modern practices rely on an instant
feedback loop with the customer. They skip lengthy design and
specification phases in favor of getting up and running as quickly
as possible. This generates feedback early allowing the developer to
deal with issues that were previously not known or not even
predictable.

The second major paradigm shift is a strong focus on testability. In
fact, testing becomes the primary goal. In the past, a feature was
considered complete when it was checked in and given a quick
demonstration. This explains how projects are said to be
feature-complete at alpha, when they still have beta and final to go
through to proof they actually do what they promise to do. As such,
the feature-complete status is a lie.

A game with twenty levels, thirty hours of gameplay time, fourty
unlockables and an unlimited number of player options is impossible
to test with black-box mechanisms. No amount of QA staff can
exhaustively cover all areas and settings of the game. Unless
engineers build all systems with testability in mind and develop
them with automated tests, it will become impossible to ship robust
products.

\subsection{On Tools}

In order to come up with better games, we should stop making them in
the first place. Because engineers are too focused on writing game
code, they don't spend enough effort on giving artists and designers
the tools and pipelines they need to do deliver the highest quality.
Because of technical limitations, engineers used to be the ones
deciding what was possible and what not. This claim to the throne
has resulted in a team chemistry that still puts too much power in
the hands of engineers. Once they realize that everybody benefits
with fast, flexible and powerful tools and pipelines, games can be
developed faster and because dedicated people can focus on what they
are best at (art, gameplay and technology) the resulting products
will be of higher quality.

\newpage
\section{Practice}

The previous two sections offered fundamental values and
observations. Explicitly talking about these is a first step, but
using them to direct our day to day focus and tasks is crucial.

This paper would be four times as long if it described everything
that could be done to stay closer to our values and act upon our
observations.

With the right people on a team, these things will bring themselves
forward almost automatically. The need for higher level languages,
component based development, more third-party technology, inter-team
sharing strategies, probability based risk management, etcetera, all
become apparent once we have values that guide us and we learn that
what is best for the team is often what is best for the product.

Nonetheless, presented below are two practical ideas that are quite
different from existing game development strategies. This makes them
worth mentioning. They have been known about for a long time in
other industries, but somehow have a hard time gaining ground within
the games industry.

\subsection{One Team, Two Products}

Most projects these days have lifetimes beyond a single cycle. That
has great implications for the development strategy because short
term solutions are not always what is best for the long term. In
essence, a franchise has to maintain a careful balance between two
conflicting goals. On the one hand its priority is to ship the best
possible game in the shortest possible time, favouring quick fixes
over solid and more durable approaches. On the other hand, it has a
responsibility to make sure that technology can be reused the next
cycle and beyond.

Traditionally, one person is in charge of both aspects of
technology. This technical lead is held accountable for providing
the technology that ships this years title and simultaneously
responsible for delivering the technology base that will shape next
years game.

Given the rush in our industry and the growing pressure to deliver
more features on tighter schedules, it becomes obvious that the two
objectives can not be successfully managed by one person. Placing
these burdens on a single technical lead results in stress levels
much beyond what is acceptable. Because shipping games is what
brings in money, the long term vision suffers. Unfortunately, the
next cycle last year's short term gains come to haunt us back
because we now work with a technology base that isn't ready. This
further increases the need for quick fixes resulting in a vicious
circle that explodes unless technology is deprecated before then.

This waste of time and effort can easily be avoided by realizing
that a franchise cycle delivers two products instead of one. First
and foremost, it is responsible for delivering a game. Secondly, and
nearly as important, it needs to finish the cycle with the
technology and strategy that is capable of an even better
development cycle than the previous.

Once the two-products paradigm is accepted, it is instantly obvious
that instead of one technical lead, two are required. One is
responsible for making sure the technology required to ship a
product is available. The other is held accountable for the
technology and strategy that ensures a smooth long term life of the
franchise.

Instead of a single technical lead with a split-personality, the two
can have healthy discussions about the trade-offs between the now
and the future. This not only avoids undue stress on a single
person, it also creates more honest and explicit decisions that can
be logged and learned from.

\subsection{Test driven development}

As explained in a previous section, a stronger focus on testing is a
necessity to ship robust next generation titles. Developers need to
write code with testing in mind, and the proven way to do this is
test-driven development. Doing this, an engineer will write a test
for a particular piece of technology even before the actual
technology itself is written. This is one step further than the
practice of writing tests after implementing a feature, which in
itself is a useful but also largely ignored development style.

The advantages of test driven development are many.

\begin{itemize}
    \item Writing the test first forces the author to think about the
    interface, exposing design issues as early as possible. The
    sooner these can be resolved the smaller their impact and the
    less time is wasted.
    \item Testing forces decoupled and independent systems, reducing build
    times, improving code flexibility and making sharing easier.
    \item Tests can act as self-documenting pieces of code
    demonstrating how to use a piece of technology.
    \item Test environments are more efficient to work in due to
    their minimal nature. Instead of having to compile and load the entire game,
    a test could depend on and initialize the bare minimum to run
    successfully.
    \item Tests are the only way to gain confidence that changes in
    the code didn't introduce new issues. Continuous testing is the earliest way to detect
    regressions. Seeing all existing tests successfully run upon a
    rewrite is the only possible method to do quality assurance.
\end{itemize}

The single most important advantage of test driven development is
how predictable projects become. All features in the game come with
tests that proof them with a high level of confidence, much higher
than what traditional development efforts can promise. This means
that by the time alpha roles around, the so called feature-complete
status really is feature complete and the resulting external quality
assurance phase can be much smoother and shorter.

\section{Conclusion}

A development strategy derived solely from feature requests and a
ship date can not be successful. We first need to understand what
motivates people to be part of a project.

People need an environment build on integrity, trust and honesty.
They want to work on exciting and difficult projects to stay
passionate and generate a sense of pride. They have a need to work
with peers that, like them, set high standards and deliver
impeccable results. This generates a team reputation that
strengthens the excitement. It also creates a dynamic that
encourages growth, which is another fundamental motivator that keeps
people focused, dedicated and productive.

Small teams of good people can be more effective than large teams of
mediocre people. Going past a threshold in either direction affects
the quality bar across the entire team, lifting it to higher levels
or dragging it further down.

Game development is not what it was ten years ago. Our projects have
more in common with other software industry projects than ever
before. The complexity of future projects will be unmanageable with
traditional practices, and we need to learn from other industries to
find new development models. Much work is necessary to create better
tools and pipelines for artists and designers. Only when they can
effectively do what they're best at, can games reach the next level.

Multi-cycle franchises ship more than just a game on every cycle.
They are also responsible for establishing a solid and continued
technology platform that largely determines the success for future
years. The two products have conflicting priorities, and require two
separate people to focus on.

Test driven development has proven to be an efficient development
model, capable of delivering robust software with a high level of
confidence. It maintains a high level of reliability throughout a
project, instantly detecting regressions. It also forces developers
to think about their goals before they work on a feature.

Most importantly, all people are great at what they do. We need to
provide the environment that allows them to do incredible things
without getting sidetracked by things that don't directly support
their values and goals. Do this, and great success becomes
inevitable.

\end{document}
s
